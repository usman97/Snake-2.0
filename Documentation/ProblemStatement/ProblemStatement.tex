\documentclass{article}

\usepackage{tabularx}
\usepackage{booktabs}
\usepackage{xcolor}

\title{SE 3XA3: Problem Statement\\ \textbf{\textcolor{red}{Snake 2.o}}}

\author{Team \#30, VUA30
		\\ Usman Irfan - irfanm7
		\\ Andy Hameed - hameea1
		\\ Vaibhav Chadha - chadhav
}

\date{2018-09-21}

\usepackage[left=2cm, right=5cm, top=2cm]{geometry}

\begin{document}

\begin{table}[hp]
\caption{Revision History} \label{TblRevisionHistory}
\begin{tabularx}{\textwidth}{llX}
\toprule
\textbf{Date} & \textbf{Developer(s)} & \textbf{Change}\\
\midrule
2018-09-20 & Vaibhav & Made the LaTeX file and wrote the section with what the Problem is\\
2018-09-20 & Usman & Added the Importance of Problem section while formatting the LaTeX\\
2018-09-21 & Andy Hameed & Formatted LaTeX file and added Context section, giving the final editing to the document\\
\textcolor{red}{2018-10-09} & Andy Hameed & Edited doc to reflect web app to desktop app decision change\\
\textcolor{red}{2018-12-04} & Andy Hameed & Small edit to problem statement to rephrase the context of the project\\
\textcolor{red}{2018-12-02} & Vaibhav Chadha & Revision one changes to make the document better.\\
\textcolor{red}{2018-12-03} & Usman Irfan & Edited Importance of the Problem\\
\bottomrule
\end{tabularx}
\end{table}

\newpage

\maketitle

\subsection*{The Problem}

Almost everyone nowadays relies on a computer as a multipurpose tool for research, video streaming, gaming and many other tasks. With the emergence of fast computing, gaming has become a popular pastime activity and a source of entertainment. \textcolor{red}{Many gamers enjoy playing classical PC games such as the the Solitaire suite of games and the Snake game.} A simple, memory efficient desktop application of the Snake game allows it to be accessible for gamers \textcolor{red}{anywhere and at any time without the requirement of internet access. Our team, VUA30, will be creating a desktop application for the well-known Snake game with new enhancements and features. This competitive and addictive game can allow the user to play at \textcolor{red}{different levels and challenge their own highscore} }

\subsection*{Importance of the Problem}

Buying a computing device with high storage and faster performance can be out of budget. Complicated software covers up all the storage and the user is bound to use these applications \textcolor{red}{without} downloading other software. The importance of the redevelopment of The Snake is to save computing \textcolor{red}{device's} personal storage and allow the user to play a game 24/7 with strong performance, \textcolor{red}{even offline}. \textcolor{red}{ Creating a desktop application of the snake game can fit into the category of classic games such as the solitaire suite. The recreation of this game will allow the user to enjoy the traditional game anytime and anywhere as long as they have installed the application. Improving aspects such as graphics and custom speed will also make the game more interesting. We would add more features to the game to make it more customizable and let people enjoy the classic game impressively.}


\subsection*{Context}
The stakeholders are mainly the gaming audience, the older generation of game enthusiasts, youth and teens. Although the game can be played by anyone, it is targeted towards the audience mentioned above who are most invested in the game. The environment is native app development using Python and the Pygame library. This provides less flexibility than web API's but there still exists many useful Python libraries which can be used such as Pygame. 
\end{document}

