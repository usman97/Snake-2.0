\documentclass[12pt, titlepage]{article}

\usepackage{booktabs}
\usepackage{tabularx}
\usepackage{hyperref}
\usepackage{amsmath}
\usepackage{xcolor}
\usepackage{url}
\usepackage{soul}

\hypersetup{
    colorlinks,
    citecolor=black,
    filecolor=black,
    linkcolor=black,
    urlcolor=blue
}
\usepackage[round]{natbib}

\title{SE 3XA3: Test Plan\\Snake 2.o}

\author{Team 30, VUA30
	\\ Andy Hameed and hameea1
	\\ Usman Irfan and irfanm7
	\\ Vaibhav Chadah and chadhav
}

\date{\today}


\begin{document}

\maketitle

\pagenumbering{roman}
\tableofcontents
\listoftables
\listoffigures

\begin{table}[bp]
\caption{\bf Revision History}
\begin{tabularx}{\textwidth}{p{3cm}p{2cm}X}
\toprule {\bf Date} & {\bf Version} & {\bf Notes}\\
\midrule
10/25/2018 & 1.0 & Usman added section 3.1, Vaibhav added section 3.2, Andy added section 4. \\
10/26/2018 & 1.0 & Andy added section 1 and 7, Vaibhav added section 2 and 5, Usman added section 6.\\
\textcolor{red}{12/2/2018} & 1.1 & Vaibhav is doing the revision 1 in order to improve the documents quality\\
\textcolor{red}{12/2/2018} & 1.1 & Usman is doing the revision 1 in order to improve the documents quality\\
\bottomrule
\end{tabularx}
\end{table}

\newpage

\pagenumbering{arabic}

\section{General Information}

\subsection{Purpose}
Testing will be conducted to ensure that the software meets the requirements set in the original development as well as the new requirements that are set as enhancements to the original game. The generated test cases will also serve as guiding rules for developing code using TDD.

The reinvention of the Snake game as Snake 2.o will involve new features such as custom speed, high score menu, customizable themes and a multiplayer mode if time permits. These new features, along with the requirements set for the original implementation of the game will be tested to detect any bugs or errors. The use of the Pygame library allows for a GUI that will enable integrated and system testing. Peers will be able to demo the game and catch any errors or bugs by doing so. White box testing will be used as well on the existing code that was used for the POC demonstration. \textcolor{red}{ \st{Automated testing will be implemented using the unittest framework built into Python}}. Aspects  that have not yet been implemented or would be hard to detect visually, like for example the speed of the moving snake, will be tested using static analysis along with automated testing.
\subsection{Scope}
Testing will cover all of the behaviours mentioned below. Note that this is a general overview and more details are provided further on in the document.

The expected behaviour is to have a menu that leads to the game screen. Once the game is initiated, the objective is for the snake to eat the food block and continue doing so until the snake dies. Each time the snake eats, the score should increase by one and the length of the snake body should increase by some predetermined number of blocks. The snake dies if it runs into itself by looping around its body or by hitting the edges of the game window. 
Looking at the Git respository for the original game, there are no test modules that can be seen so the test cases will be based on any test cases generated by the three team members.

\subsection{Acronyms, Abbreviations, and Symbols}
	
\begin{table}[hbp]
\caption{\textbf{Table of Abbreviations}} \label{Table}

\begin{tabularx}{\textwidth}{p{3cm}X}
\toprule
\textbf{Abbreviation} & \textbf{Definition} \\
\midrule
T1D1 & Test 1 ID 1\\
TDD & Test-Driven Development\\
POC & Proof of Concept\\
GUI & Graphical User Interface\\
N/A & Not Applicable\\
\bottomrule
\end{tabularx}

\end{table}

\begin{table}[!htbp]
\caption{\textbf{Table of Definitions}} \label{Table}

\begin{tabularx}{\textwidth}{p{3cm}X}
\toprule
\textbf{Term} & \textbf{Definition}\\
\midrule
Pygame & open source Python library used to create game graphics\\
White Box testing & a method of testing where the code is examined in order to create the test cases. This mostly corresponds to testing functional requirements based on the description of functions and methods in each component module\\
Black Box testing & a method of testing where the code is not examined in order to create the test cases. This mostly corresponds to non-functional requirements\\
Test Driven Development& a method of developing software using a set of test cases that are written prior to the code itself. The test cases can identify how functions and methods should behave\\
PyUnit& Testing framework for python software development\\
\bottomrule
\end{tabularx}

\end{table}	

\subsection{Overview of Document}

The document will summarize the test cases that will be conducted on Snake 2.o, a remake of the orignal snake game using Python and the Pygame library. Several testing techniques are used including automated testing, white box testing, black box testing, manual testing, integration and system testing and static analysis. The document will outline the plan for testing, a description of the test system with non-functional and functional test cases, unit testing and POC testing, and other details pertaining to the testing of Snake 2.o.

\section{Plan}

\subsection{Software Description}

The software will act as a medium of entertainment to the users. It is a snake game with added functionality such as different speed and theme options. The implementation of this software is done using Python. 

\subsection{Test Team}

The individuals responsible for testing are Vaibhav Chadha, Usman Irfan and Andy Hameed. Each person will be responsible for testing one's own work. For example, Vaibhav is working on the Graphical User interface of the main screen, hence is responsible for testing it. Usman and Andy will be collaboratively working on the snake game ( which includes recording highest score, current score, snake movement etc.) and will be responsible for testing them likewise.

\subsection{Automated Testing Approach}

One of the more difficult parts of testing the software will be manual testing. The reason behind this is that a game can be tested better when played as the user can see errors and delays better.

\st{However, automated testing will also be done in order to check certain functionality of the software. For this, PyUnit testing will be used.}

\subsection{Testing Tools}

PyUnit testing will be used as a testing tool for this program. 

\subsection{Testing Schedule}

See \href{https://gitlab.cas.mcmaster.ca/hameea1/se3xa3/tree/master/BlankProjectTemplate/ProjectSchedule}{Gantt Chart} for details about the testing schedule
	
\section{System Test Description}
	
\subsection{Tests for Functional Requirements}
		
\begin{enumerate}

\subsubsection{Testing Functions \& Methods}
\item{\textbf{TID1}\\}

\textcolor{red} {Type: Functional, Dynamic, manual\\
Initial State: The desktop application starts waiting for the user to select a game mode to begin.\\
Input: The user selects one of the game modes from the main menu.\\
Output: The desktop application moves to the next page by displaying the themes.\\
How the test will be performed: The test will be done dynamically, that means once the program will be executed the tester would select a difficulty level and see if the page gets updated by displaying different themes.}

\item{\textbf{TID2}\\}

Type: Functional, Dynamic, Manual
					
\textcolor{red} {Type: Functional, Dynamic, manual\\
Initial State: The desktop application is at the theme mode and requires user input to select an input and display the game mode.\\
Input: The user selects one of the themes from the theme page interface.\\
Output: The desktop application moves to the next page by displaying the gameplay interface.\\
How the test will be performed: The test will be done dynamically, that means once the program will be executed the tester would select a theme and see if the page gets updated by displaying the gameplay.}


\item{\textbf{TID3}\\}

Type: Functional, Dynamic, Manual
					
\textcolor{red} {Type: Functional, Dynamic, manual\\
Initial State: The desktop application executes and displays a ``Highscore" button on the bottom left of the main menu.\\
Input: The user selects the ``Highscore" button.Output: The desktop application moves to the next page by displaying the gameplay interface.\\}

Output: The application would display the highest score of the user from the day they started to play till the present date.

How the test will be performed: The test will be done dynamically, that means once the program will be executed the tester would select a theme and see if the page gets updated by displaying the gameplay.


\item{\textbf{TID4}\\}
Type: Functional, Dynamic, manual

Initial State: The desktop application starts waiting for the user to enter a command to begin.

Input: The user presses any button key.

Output: The desktop application begins moving the snake towards the respective direction.

How the test will be performed: The test will be done dynamically, that means once the program will be executed the developer will press any key to see if it would run the game, making the snake move.

\item{\textbf{TID5}\\}

Type: Functional, Dynamic, Manual 	
					
Initial State: The desktop application executes and displays the snake at a random location.	
				
Input: NULL

Output: The snake displays the snake at random location when played the next time.
					
How test will be performed: The user can track the location of the snake the first time the game is played. The game should be restarted to ensure that the snake's position changes every time the game starts.

\item{\textbf{TID6}\\}

Type: Functional, Dynamic, Manual 	
					
Initial State: The snake's food is at a random location.
				
Input: NULL.

Output: The food reappears on the screen at a random location when the snakes eat the previous one.
        		
How test will be performed: The developer will test this requirement by moving the snake's head location equal to the food's location. When the snake eats the food, instantly another food should display on the screen at a random location. 

\subsubsection{Testing of Keyboard/Mouse Interactions}

\item{\textbf{TID7}\\}

Type:Functional, Dynamic, Manual 	
					
Initial State: The game waits for the user to press a direction key to move the snake.
					
Input: The user presses UP key.
					
Output: The snake in the game would moves up by one-unit length.
					
How test will be performed: The test will be done dynamically, that means once the program will be executed the developer will press the UP key to test if the snake moves in the upward direction.

\item{\textbf{TID8}\\}

Type: Functional, Dynamic, Manual 	
					
Initial State: The game waits for the user to press a direction key to move the snake.
					
Input: The user presses DOWN key.
					
Output: The snake in the game would moves down by one-unit length.
					
How test will be performed: The test will be done dynamically, that means once the program will be executed the developer will press the DOWN key to test if the snake moves in the downward direction.


\item{\textbf{TID9}\\}

Type: Functional, Dynamic, Manual 	
					
Initial State: The game waits for the user to press a direction key to move the snake.
					
Input: The user presses LEFT key.
					
Output: The snake in the game would moves left by one-unit length.
					
How test will be performed: The test will be done dynamically, that means once the program will be executed the developer will press the LEFT key to test if the snake moves in the left direction.

\item{\textbf{TID10}\\}

Type: Functional, Dynamic, Manual 	
					
Initial State: The game waits for the user to press a direction key to move the snake.
					
Input: The user presses RIGHT key.
					
Output: The snake in the game would moves right by one-unit length.
					
How test will be performed: The test will be done dynamically, that means once the program will be executed the developer will press the RIGHT key to test if the snake moves in the right direction.

\item{\textbf{TID11}\\}

Type: Functional, Dynamic, Manual

Initial State: The desktop application executes and displays three modes to be played.

Input: Mouse Cursor

Output: The application should open the specific mode the user has requested to play.

How test will be performed: Different modes in the game will be opened using the mouse cursor, their display or speed should be different from other modes. Easy having the slowest speed and allowing the snake to exit from the one-direction boundary and enter from the other direction of the boundary (e.g. leaving from right side boundary and entering from the left side boundary). While playing the hard mode, the speed should be much faster than the Easy mode, and would not allow the snake to cross the boundary. If the snake touches the boundary the snake should die and terminating the game.


\item{\textbf{TID12}\\}

Type: Functional, Dynamic, Manual 	
					
Initial State: The initial length of the snake would be one-unit length.
				
Input: The user presses the Direction keys to control the snake

Output: The length of the snake should not equal to one-unit length when it dies (Hard mode would be an exception).
					
How test will be performed: The developer moves the snake by pressing the direction keys. When the snake's head location equals the food location, its length should be increased by five unit-length. When the snake dies its increase in length      \textcolor{red}{its length should be increased by three unit-length}.


\subsubsection{Testing Game Ending}
\item{\textbf{TID13}\\}
Type: Functional, Dynamic, Manual 	
					
Initial State: The snake is not one-unit length.
				
Input: NULL.\\
\textcolor{red}{Output: The screen displays a screen biting itself, and a message prompts on the screen with various options to allow the user to play the game again or go to the main menu.\\
How the test will be performed: The developer will test this requirement by moving the snake’s head location to the snake’s body location. When the snake eats its body the snake’s movement should stop and will be able to see the error message.}
\end{enumerate}


\subsection{Tests for Nonfunctional Requirements}

\subsubsection{Look and Feel}

\begin{enumerate}
	
	\item{\textbf{TID14}\\}
	
	Type: Structural, Dynamic, Manual
	
	Initial State: The game should be installed on the device.
	
	Input/Condition: The game is opened and ran on the device.
	
	Output/Result: The User Interface should open with different buttons, alongside with a playground with a snake in it.
	
	How the test will be performed: The program will manually run on the device and checked by the human eye to see if it meets the criteria. 
	
	
\end{enumerate}

\subsubsection{Usability}

\begin{enumerate}
	
	\item{\textbf{TID15}\\}
	
	Type: Structural, Dynamic, Manual
	
	Initial State: The program will be running for a human nearing 10 years of age or above
	
	Input/Condition: The program will be set on its default settings
	
	Output/Result: The person testing should be able to understand the game and play it. He/she should be able to customize themes and speed of the game.
	
	How the test will be performed: A younger human of nearing age 10 will be asked to operate this game and recorded if he/she is able to operate it successfully or not. 
	
	
\end{enumerate}

\subsubsection{Performance}

\begin{enumerate}
	
	\item{\textbf{TID16}\\}
	
	Type:  Functional, Dynamic, Manual
	
	Initial State: The program will be running with the main user interface open.
	
	Input/Condition: The button is pressed.
	
	Output/Result: The response time for button should be less than half a second.
	
	How the test will be performed: It will be performed using human actions. The response would be times to be as precise as possible. Also, it will be taken into consideration that the user doesn't have to wait for a long observable time.
	
	\item{\textbf{TID17}\\}
	
	Type: Structural, Dynamic, Manual 
	
	Initial State: The snake Game will be running on the device.
	
	Input/Condition: \textcolor{red}{Different modes will be played}.
	
	Output/Result: \textcolor{red}{Snake's speed variation will be observed between different modes.}
	
	How the test will be performed: \textcolor{red}{The game will be played with selecting 1 of the 3 different levels. Then, the speed difference will be observed as the game progresses through. Also, it will be taken care that the game goes at constant speed within a specific level.}
	
	\item{\textbf{TID18}\\}
	
	Type: Structural, Dynamic, Manual 
	
	Initial State: The snake Game will be running on the device.
	
	Input/Condition: The snake will be moving around and keys will be pressed to change directions.
	
	Output/Result: Snake should change directions promptly.
	
	How the test will be performed: While the game is going on, the buttons will be pressed to change the direction of the snake.
	
	
\end{enumerate}

\subsubsection{Operational and Environmental}

\begin{enumerate}
	
	\item{\textbf{TID19}\\}
	
	Type: Structural, Dynamic, Manual
	
	Initial State: The program will be moved on a USB.
	
	Input/Condition: The USB will be inserted into any other working computer/ Desktop.
	
	Output/Result: The game should be able to run on it as long as the device is powered and in working state.
	
	How test will be performed: Many different laptops, alongside with desktops, will be used to test. The game will be played on different devices with different specifications to make sure that the game is playable regardless of the specs of the device.
	
	
\end{enumerate}

\subsubsection{Maintainability and Support Requirements}

\begin{enumerate}
	
	\item{\textbf{TID20}\\}
	
	Type: Functiona Dynamic, Manual
	
	Initial State: The program will be moved to a Windows, Mac OS and Linux operating devices.
	
	Input/Condition: The program will be executed.
	
	Output/Result: The game should run.
	
	How test will be performed: The game will be taken and transferred to the systems operating on different OS's. For this, the target is Windows device, Mac OS device and a Linux Device. 
	
	
\end{enumerate}

\subsubsection{Security and Cultural}

\begin{enumerate}
	
	\item{\textbf{TID21}\\}
	
	Type:  Structural, Dynamic, Manual
	
	Initial State: The program will be running.
	
	Input/Condition: All the interfaces running.
	
	Output/Result: No offensive or illegal content on the entire application.
	
	How test will be performed: The application will be executed and each page and option will be approached to make sure there is no offensive or illegal content. Also, there is a Static module to this requirement where all the files (including code) will be looked to make sure about no offensive or illegal content.
	
\end{enumerate}

\section{Tests for Proof of Concept}

The POC consists of a simple demonstration of the moving snake in the game window along with a start menu. The food item will not be created in the demo, instead, the testing will only involve the movement of the snake and the main menu that has been created at the start of the game.

\subsection{Snake Dynamics}
		
\paragraph{Snake Movement and Speed}

\begin{enumerate}

\item{\textbf{TID25}\\}

Type: Dynamic
					
Initial State: The snake body - graphically represented by a red square - is initially motionless. It exists somewhere within the frame of the window.
					
Input: Keyboard Event - user clicks on one of the directions on the keyboard arrow pad.
					
Output: Snake moves according to the direction chosen. This can logically represented by the expression keyboardEvent.direction == snakeMovementDirection. Note that the variables used are arbitrary and are dependant on Python syntax.
					
How test will be performed:
\begin{itemize}
\item A method will be created under the POC test class where the keyboard event is manually set to the code representing each of the directions on the arrow keypad - up, down, left and right. The direction inserted will be asserted equal to the direction of the moving snake, set by some variable.
\item After starting the game, the user will click on each one of the four directions and verify whether or not the snake is moving in the corresponding direction, using the graphical interface created with Pygame.
\end{itemize}
			
\item{\textbf{TID26}\\}

Type: Dynamic, Functional testing
					
Initial State: The snake body - graphically represented by a red square - is initially motionless. It exists somewhere within the frame of the window.
					
Input: Keyboard Event - user clicks on one of the directions on the keyboard arrow pad.  
					
Output: Snake moves accurately according to the speed set in the snake module. Statically, this can represented for the vertical movement of the snake by this expression:
\begin{align*}
	(snakeFinalPosition - snakeInitPosition) == (speed \times timeElapsed)*vel
\end{align*}
 where vel is the distance defined for 1 single step and speed is the delay between each step in milliseconds		

			
How test will be performed: A method will be created under the POC test class where the keyboard event is manually set to the code representing each of the directions on the arrow keypad - up, down, left and right. The logical expression above is implemented into an assert statement verifying that the distance moved corresponds to the speed and velocity that were used as well as the time that has elapsed - this can be obtained from the time object in Pygame.


\end{enumerate}

\subsection{Integration and System Testing}

\begin{enumerate}

\item{\textbf{TID27}\\}

Type: Integration, System testing
					
Initial State: The game menu is loaded onto the window with options for starting the game and quitting the game.
					
Input: The following sequence of inputs

\begin{enumerate}
\item User clicks start game
\item  Keyboard Event - user clicks on one of the directions on the keyboard arrow pad.
\item user exits the window by clicking the exit tab on the top right corner of the screen  
\end{enumerate}					

Output: Snake game runs as intended. The "start game" option leads the user to the game screen where the snake body sits motionless. The user moves the snake body using the keyboard arrow pad, moving in directions that correspond apprioriately to the arrows clicked and in the correct sequence. The game window is closed once the user clicks the exit button on the top right corner.	
			
How test will be performed: Several peers will be asked to test the game from start to finish for this integration and system test.


\end{enumerate}
	
\section{Comparison to Existing Implementation}	
				
Currently, we have the following tests that compare to the existing comparison:

\begin{enumerate}
	\item TID3 : In the Proof of Concept, it has already been tested that the snake appears at a random position everytime a new game is played.
	
	\item TID6, TID7, TID8, TID9 : In the code, its already tested that the snake moves in the direction of the button pressed as soon as it is pressed. 
	
	\item TID19 : The current code for the game meets the requirement as it launched the operation of a button as soon as it is pressed. For this, the "Play Game" button and "Quit" button has been tested.
	
	\item TID23 : The present code was transferred on windows and Mac OS devices and was working completely fine.
	
	\item TID25 : All the mentioned testing has been done for the POC.
\end{enumerate}

\section{Unit Testing Plan}
\st{The PyUnit testing framework would be used to test our desktop application.
\subsection{Unit testing of internal functions}
The PyUnit testing framework will be used to test our source code's functions, this is an automated testing unit, and it provides classes which can ease different testing functions. By using PyUnit we can check the robustness of our program, if wrong inputs are given will the program be able to handle such cases without crashing. Besides, the requirement of the program can be tested to see if our program matches with the functional and non-functional requirements of the program.}
\subsection{Unit testing of output files}		
\st{The testing of the output files through unit testing will tell the developers if all the test cases designed by them run efficiently. The snake's movement would be compared to the actual output if the user is pressing the UP key and the snake is moving in the respective direction it would pass the unit testing. Testing the output files can also help us to find that if different modes of the game are selected then different rules of the games should be followed. The game being played in the Hard mode could be tested that the snake is not allowed to cross boundaries and this could be compared with automated testing allowing us to know if our output files have passed their unit test.}

Due to the nature of the project, manual and integrated/system testing will be performed throughout the development process to check for correctness and verify adherence to functional and nonfunctional requirements.


\newpage

\section{Appendix}

\subsection{Symbolic Parameters}

N/A

\subsection{Usability Survey Questions?}

The following questions will be asked to peers when conducting integrated and system testing:
\begin{itemize}
\item Does the game lag at any point?
\item Does the game maintain consistent speed performance as you advance through the game?
\item Is the main menu clear and understandable?
\item Is the game exciting to play?
\item Is the game visually appealing? Does it look visually complete?
\end{itemize}
Other questions will be asked to validate the game, mostly focusing on non-functional requirements whose completeness is subjective to the user.

\end{document}